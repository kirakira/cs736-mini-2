%% bare_conf.tex
%% V1.3
%% 2007/01/11
%% by Michael Shell
%% See:
%% http://www.michaelshell.org/
%% for current contact information.
%%
%% This is a skeleton file demonstrating the use of IEEEtran.cls
%% (requires IEEEtran.cls version 1.7 or later) with an IEEE conference paper.
%%
%% Support sites:
%% http://www.michaelshell.org/tex/ieeetran/
%% http://www.ctan.org/tex-archive/macros/latex/contrib/IEEEtran/
%% and
%% http://www.ieee.org/

%%*************************************************************************
%% Legal Notice:
%% This code is offered as-is without any warranty either expressed or
%% implied; without even the implied warranty of MERCHANTABILITY or
%% FITNESS FOR A PARTICULAR PURPOSE! 
%% User assumes all risk.
%% In no event shall IEEE or any contributor to this code be liable for
%% any damages or losses, including, but not limited to, incidental,
%% consequential, or any other damages, resulting from the use or misuse
%% of any information contained here.
%%
%% All comments are the opinions of their respective authors and are not
%% necessarily endorsed by the IEEE.
%%
%% This work is distributed under the LaTeX Project Public License (LPPL)
%% ( http://www.latex-project.org/ ) version 1.3, and may be freely used,
%% distributed and modified. A copy of the LPPL, version 1.3, is included
%% in the base LaTeX documentation of all distributions of LaTeX released
%% 2003/12/01 or later.
%% Retain all contribution notices and credits.
%% ** Modified files should be clearly indicated as such, including  **
%% ** renaming them and changing author support contact information. **
%%
%% File list of work: IEEEtran.cls, IEEEtran_HOWTO.pdf, bare_adv.tex,
%%                    bare_conf.tex, bare_jrnl.tex, bare_jrnl_compsoc.tex
%%*************************************************************************

% *** Authors should verify (and, if needed, correct) their LaTeX system  ***
% *** with the testflow diagnostic prior to trusting their LaTeX platform ***
% *** with production work. IEEE's font choices can trigger bugs that do  ***
% *** not appear when using other class files.                            ***
% The testflow support page is at:
% http://www.michaelshell.org/tex/testflow/



% Note that the a4paper option is mainly intended so that authors in
% countries using A4 can easily print to A4 and see how their papers will
% look in print - the typesetting of the document will not typically be
% affected with changes in paper size (but the bottom and side margins will).
% Use the testflow package mentioned above to verify correct handling of
% both paper sizes by the user's LaTeX system.
%
% Also note that the "draftcls" or "draftclsnofoot", not "draft", option
% should be used if it is desired that the figures are to be displayed in
% draft mode.
%
\documentclass[11pt,conference]{IEEEtran}
% Add the compsoc option for Computer Society conferences.
%
% If IEEEtran.cls has not been installed into the LaTeX system files,
% manually specify the path to it like:
% \documentclass[conference]{../sty/IEEEtran}





% Some very useful LaTeX packages include:
% (uncomment the ones you want to load)


% *** MISC UTILITY PACKAGES ***
%
%\usepackage{ifpdf}
% Heiko Oberdiek's ifpdf.sty is very useful if you need conditional
% compilation based on whether the output is pdf or dvi.
% usage:
% \ifpdf
%   % pdf code
% \else
%   % dvi code
% \fi
% The latest version of ifpdf.sty can be obtained from:
% http://www.ctan.org/tex-archive/macros/latex/contrib/oberdiek/
% Also, note that IEEEtran.cls V1.7 and later provides a builtin
% \ifCLASSINFOpdf conditional that works the same way.
% When switching from latex to pdflatex and vice-versa, the compiler may
% have to be run twice to clear warning/error messages.






% *** CITATION PACKAGES ***
%
%\usepackage{cite}
% cite.sty was written by Donald Arseneau
% V1.6 and later of IEEEtran pre-defines the format of the cite.sty package
% \cite{} output to follow that of IEEE. Loading the cite package will
% result in citation numbers being automatically sorted and properly
% "compressed/ranged". e.g., [1], [9], [2], [7], [5], [6] without using
% cite.sty will become [1], [2], [5]--[7], [9] using cite.sty. cite.sty's
% \cite will automatically add leading space, if needed. Use cite.sty's
% noadjust option (cite.sty V3.8 and later) if you want to turn this off.
% cite.sty is already installed on most LaTeX systems. Be sure and use
% version 4.0 (2003-05-27) and later if using hyperref.sty. cite.sty does
% not currently provide for hyperlinked citations.
% The latest version can be obtained at:
% http://www.ctan.org/tex-archive/macros/latex/contrib/cite/
% The documentation is contained in the cite.sty file itself.






% *** GRAPHICS RELATED PACKAGES ***
%
\ifCLASSINFOpdf
  % \usepackage[pdftex]{graphicx}
  % declare the path(s) where your graphic files are
  % \graphicspath{{../pdf/}{../jpeg/}}
  % and their extensions so you won't have to specify these with
  % every instance of \includegraphics
  % \DeclareGraphicsExtensions{.pdf,.jpeg,.png}
\else
  % or other class option (dvipsone, dvipdf, if not using dvips). graphicx
  % will default to the driver specified in the system graphics.cfg if no
  % driver is specified.
  % \usepackage[dvips]{graphicx}
  % declare the path(s) where your graphic files are
  % \graphicspath{{../eps/}}
  % and their extensions so you won't have to specify these with
  % every instance of \includegraphics
  % \DeclareGraphicsExtensions{.eps}
\fi
% graphicx was written by David Carlisle and Sebastian Rahtz. It is
% required if you want graphics, photos, etc. graphicx.sty is already
% installed on most LaTeX systems. The latest version and documentation can
% be obtained at: 
% http://www.ctan.org/tex-archive/macros/latex/required/graphics/
% Another good source of documentation is "Using Imported Graphics in
% LaTeX2e" by Keith Reckdahl which can be found as epslatex.ps or
% epslatex.pdf at: http://www.ctan.org/tex-archive/info/
%
% latex, and pdflatex in dvi mode, support graphics in encapsulated
% postscript (.eps) format. pdflatex in pdf mode supports graphics
% in .pdf, .jpeg, .png and .mps (metapost) formats. Users should ensure
% that all non-photo figures use a vector format (.eps, .pdf, .mps) and
% not a bitmapped formats (.jpeg, .png). IEEE frowns on bitmapped formats
% which can result in "jaggedy"/blurry rendering of lines and letters as
% well as large increases in file sizes.
%
% You can find documentation about the pdfTeX application at:
% http://www.tug.org/applications/pdftex





% *** MATH PACKAGES ***
%
%\usepackage[cmex10]{amsmath}
% A popular package from the American Mathematical Society that provides
% many useful and powerful commands for dealing with mathematics. If using
% it, be sure to load this package with the cmex10 option to ensure that
% only type 1 fonts will utilized at all point sizes. Without this option,
% it is possible that some math symbols, particularly those within
% footnotes, will be rendered in bitmap form which will result in a
% document that can not be IEEE Xplore compliant!
%
% Also, note that the amsmath package sets \interdisplaylinepenalty to 10000
% thus preventing page breaks from occurring within multiline equations. Use:
%\interdisplaylinepenalty=2500
% after loading amsmath to restore such page breaks as IEEEtran.cls normally
% does. amsmath.sty is already installed on most LaTeX systems. The latest
% version and documentation can be obtained at:
% http://www.ctan.org/tex-archive/macros/latex/required/amslatex/math/





% *** SPECIALIZED LIST PACKAGES ***
%
%\usepackage{algorithmic}
% algorithmic.sty was written by Peter Williams and Rogerio Brito.
% This package provides an algorithmic environment fo describing algorithms.
% You can use the algorithmic environment in-text or within a figure
% environment to provide for a floating algorithm. Do NOT use the algorithm
% floating environment provided by algorithm.sty (by the same authors) or
% algorithm2e.sty (by Christophe Fiorio) as IEEE does not use dedicated
% algorithm float types and packages that provide these will not provide
% correct IEEE style captions. The latest version and documentation of
% algorithmic.sty can be obtained at:
% http://www.ctan.org/tex-archive/macros/latex/contrib/algorithms/
% There is also a support site at:
% http://algorithms.berlios.de/index.html
% Also of interest may be the (relatively newer and more customizable)
% algorithmicx.sty package by Szasz Janos:
% http://www.ctan.org/tex-archive/macros/latex/contrib/algorithmicx/




% *** ALIGNMENT PACKAGES ***
%
%\usepackage{array}
% Frank Mittelbach's and David Carlisle's array.sty patches and improves
% the standard LaTeX2e array and tabular environments to provide better
% appearance and additional user controls. As the default LaTeX2e table
% generation code is lacking to the point of almost being broken with
% respect to the quality of the end results, all users are strongly
% advised to use an enhanced (at the very least that provided by array.sty)
% set of table tools. array.sty is already installed on most systems. The
% latest version and documentation can be obtained at:
% http://www.ctan.org/tex-archive/macros/latex/required/tools/


%\usepackage{mdwmath}
%\usepackage{mdwtab}
% Also highly recommended is Mark Wooding's extremely powerful MDW tools,
% especially mdwmath.sty and mdwtab.sty which are used to format equations
% and tables, respectively. The MDWtools set is already installed on most
% LaTeX systems. The lastest version and documentation is available at:
% http://www.ctan.org/tex-archive/macros/latex/contrib/mdwtools/


% IEEEtran contains the IEEEeqnarray family of commands that can be used to
% generate multiline equations as well as matrices, tables, etc., of high
% quality.


%\usepackage{eqparbox}
% Also of notable interest is Scott Pakin's eqparbox package for creating
% (automatically sized) equal width boxes - aka "natural width parboxes".
% Available at:
% http://www.ctan.org/tex-archive/macros/latex/contrib/eqparbox/





% *** SUBFIGURE PACKAGES ***
%\usepackage[tight,footnotesize]{subfigure}
% subfigure.sty was written by Steven Douglas Cochran. This package makes it
% easy to put subfigures in your figures. e.g., "Figure 1a and 1b". For IEEE
% work, it is a good idea to load it with the tight package option to reduce
% the amount of white space around the subfigures. subfigure.sty is already
% installed on most LaTeX systems. The latest version and documentation can
% be obtained at:
% http://www.ctan.org/tex-archive/obsolete/macros/latex/contrib/subfigure/
% subfigure.sty has been superceeded by subfig.sty.



%\usepackage[caption=false]{caption}
%\usepackage[font=footnotesize]{subfig}
% subfig.sty, also written by Steven Douglas Cochran, is the modern
% replacement for subfigure.sty. However, subfig.sty requires and
% automatically loads Axel Sommerfeldt's caption.sty which will override
% IEEEtran.cls handling of captions and this will result in nonIEEE style
% figure/table captions. To prevent this problem, be sure and preload
% caption.sty with its "caption=false" package option. This is will preserve
% IEEEtran.cls handing of captions. Version 1.3 (2005/06/28) and later 
% (recommended due to many improvements over 1.2) of subfig.sty supports
% the caption=false option directly:
%\usepackage[caption=false,font=footnotesize]{subfig}
%
% The latest version and documentation can be obtained at:
% http://www.ctan.org/tex-archive/macros/latex/contrib/subfig/
% The latest version and documentation of caption.sty can be obtained at:
% http://www.ctan.org/tex-archive/macros/latex/contrib/caption/




% *** FLOAT PACKAGES ***
%
%\usepackage{fixltx2e}
% fixltx2e, the successor to the earlier fix2col.sty, was written by
% Frank Mittelbach and David Carlisle. This package corrects a few problems
% in the LaTeX2e kernel, the most notable of which is that in current
% LaTeX2e releases, the ordering of single and double column floats is not
% guaranteed to be preserved. Thus, an unpatched LaTeX2e can allow a
% single column figure to be placed prior to an earlier double column
% figure. The latest version and documentation can be found at:
% http://www.ctan.org/tex-archive/macros/latex/base/



%\usepackage{stfloats}
% stfloats.sty was written by Sigitas Tolusis. This package gives LaTeX2e
% the ability to do double column floats at the bottom of the page as well
% as the top. (e.g., "\begin{figure*}[!b]" is not normally possible in
% LaTeX2e). It also provides a command:
%\fnbelowfloat
% to enable the placement of footnotes below bottom floats (the standard
% LaTeX2e kernel puts them above bottom floats). This is an invasive package
% which rewrites many portions of the LaTeX2e float routines. It may not work
% with other packages that modify the LaTeX2e float routines. The latest
% version and documentation can be obtained at:
% http://www.ctan.org/tex-archive/macros/latex/contrib/sttools/
% Documentation is contained in the stfloats.sty comments as well as in the
% presfull.pdf file. Do not use the stfloats baselinefloat ability as IEEE
% does not allow \baselineskip to stretch. Authors submitting work to the
% IEEE should note that IEEE rarely uses double column equations and
% that authors should try to avoid such use. Do not be tempted to use the
% cuted.sty or midfloat.sty packages (also by Sigitas Tolusis) as IEEE does
% not format its papers in such ways.





% *** PDF, URL AND HYPERLINK PACKAGES ***
%
\usepackage{url}
% url.sty was written by Donald Arseneau. It provides better support for
% handling and breaking URLs. url.sty is already installed on most LaTeX
% systems. The latest version can be obtained at:
% http://www.ctan.org/tex-archive/macros/latex/contrib/misc/
% Read the url.sty source comments for usage information. Basically,
% \url{my_url_here}.





% *** Do not adjust lengths that control margins, column widths, etc. ***
% *** Do not use packages that alter fonts (such as pslatex).         ***
% There should be no need to do such things with IEEEtran.cls V1.6 and later.
% (Unless specifically asked to do so by the journal or conference you plan
% to submit to, of course. )


% correct bad hyphenation here
\hyphenation{op-tical net-works semi-conduc-tor}


\begin{document}
%
% paper title
% can use linebreaks \\ within to get better formatting as desired
\title{Bare Demo of IEEEtran.cls for Conferences}


% author names and affiliations
% use a multiple column layout for up to three different
% affiliations
\author{\IEEEauthorblockN{Yiran Wang}
\IEEEauthorblockA{Computer Sciences\\
University of Wisconsin - Madison\\
yiran@cs.wisc.edu}
\and
\IEEEauthorblockN{Menghui Wang}
\IEEEauthorblockA{Computer Sciences\\
University of Wisconsin - Madison\\
menghui@cs.wisc.edu}}

% conference papers do not typically use \thanks and this command
% is locked out in conference mode. If really needed, such as for
% the acknowledgment of grants, issue a \IEEEoverridecommandlockouts
% after \documentclass

% for over three affiliations, or if they all won't fit within the width
% of the page, use this alternative format:
% 
%\author{\IEEEauthorblockN{Michael Shell\IEEEauthorrefmark{1},
%Homer Simpson\IEEEauthorrefmark{2},
%James Kirk\IEEEauthorrefmark{3}, 
%Montgomery Scott\IEEEauthorrefmark{3} and
%Eldon Tyrell\IEEEauthorrefmark{4}}
%\IEEEauthorblockA{\IEEEauthorrefmark{1}School of Electrical and Computer Engineering\\
%Georgia Institute of Technology,
%Atlanta, Georgia 30332--0250\\ Email: see http://www.michaelshell.org/contact.html}
%\IEEEauthorblockA{\IEEEauthorrefmark{2}Twentieth Century Fox, Springfield, USA\\
%Email: homer@thesimpsons.com}
%\IEEEauthorblockA{\IEEEauthorrefmark{3}Starfleet Academy, San Francisco, California 96678-2391\\
%Telephone: (800) 555--1212, Fax: (888) 555--1212}
%\IEEEauthorblockA{\IEEEauthorrefmark{4}Tyrell Inc., 123 Replicant Street, Los Angeles, California 90210--4321}}




% use for special paper notices
%\IEEEspecialpapernotice{(Invited Paper)}




% make the title area
\maketitle


\begin{abstract}
%\boldmath
The abstract goes here.
\end{abstract}
% IEEEtran.cls defaults to using nonbold math in the Abstract.
% This preserves the distinction between vectors and scalars. However,
% if the conference you are submitting to favors bold math in the abstract,
% then you can use LaTeX's standard command \boldmath at the very start
% of the abstract to achieve this. Many IEEE journals/conferences frown on
% math in the abstract anyway.

% no keywords




% For peer review papers, you can put extra information on the cover
% page as needed:
% \ifCLASSOPTIONpeerreview
% \begin{center} \bfseries EDICS Category: 3-BBND \end{center}
% \fi
%
% For peerreview papers, this IEEEtran command inserts a page break and
% creates the second title. It will be ignored for other modes.
\IEEEpeerreviewmaketitle

\section{Introduction}
Linux, as well as other most popular operating systems in the world, gives each running process a private address space.
The memory space of one process is completely invisible to other processes.
This simple design provides protection for free, however it also lacks the flexibility to share data among processes, compared to a shared address space architecture like what is used by Opal \cite{opal}.
For applications where data has to be shared across processes, Linux provides many ways to achieve this.
Among those, the three most popular way of sharing data across processes is \texttt{pipe}, inet socket, and sharing memory via \texttt{mmap}.
As interprocess communications are so widely used by programs and being such a fundamental mechanism provided by the operating system, its performance matters a lot.
In this report, we measure and compare the latencies and throughputs of the three interprocess communication mechanisms.

\texttt{pipe} is a system call provided by Linux.
It allows a process to create a unidirectional communication channel.
Message delivery on this channel is guaranteed to be in-order.
Programs can read and write this on this channel just like reading and writing regular files.
\texttt{pipe} is usually used together with \texttt{fork} as a interprocess communication method.
A typical use would be, the parent process first create two pipes by calling \texttt{pipe} twice, one is from parent to child, and the other is from child to parent.
Each call to \texttt{pipe} will return two file descriptors, one is for the reading of the pipe, and the other is for the writing of the pipe.
Then the parent creates the child process by calling \texttt{fork}.
After calling \texttt{fork}, the child process will inherit all opened file descriptors, in particular the $4$ file descriptors created by \texttt{pipe}.
Finally the two processes can communicate with each other by calling \texttt{read} and \texttt{write} on the two pipes.

TCP/IP protocol is a prevailing network protocol which guarantees reliability and provides many other desirable features.
Inet socket is a set of interface provided by Linux for programs to use the TCP/IP protocol.
Like \texttt{pipe}, TCP/IP is also a stream protocol, which means messages will be delivered in the order they are sent.
Though usually people use TCP to connect machines over a network, it can also be used by two processes on the same computer.
This would making inet socket another means of interprocess communication mechanism.

The previous two mechanism are all based on message passing.
The \texttt{mmap} provides interprocess communication by directly sharing memory between processes.
\texttt{mmap} is a system call that can map the contents of a file to a block of memory, so that any modification to the memory will be reflected on the file.
Linux can also create a block of mapped memory not backed by any file.
In this case, we can call \texttt{mmap} followed by a call to \texttt{fork} to create a block of memory that is to be shared between the parent process and the child process.
Writes by one process will be directly seen by another process.
However proper synchronization must be done in order for the two processes to communicate.

As these three methods are fundamentally different, we expect their performance on different workloads to be different as well.
In this paper we will explore the strengths and weakness in performance of each of the three mechanisms by measure their latencies and throughput on different workloads, and analyze the experiment results.
We will further investigate the reasons behind experimental data.

The rest of this paper is organized as follows.
In Section~\ref{sec:overview} we will give a high-level overview of our our experiments, including our objectives and how we are going to achieve them.
We will also talk about our hypothesis about the results of our experiments.
In Section~\ref{sec:clocks} we will talk about how we find a best clock for us to measure the times.
This is very important because all our data will be measured in terms of time, and if we do not have a good timer, none of the data can be trusted.
In Section~\ref{sec:method} we will explain in detail how we did our experiments.
In Section~\ref{sec:results} we will demonstrate our experiment results and draw conclusions.

\section{Overview}
\label{sec:overview}
The purpose of our experiments is to evaluate the performance of the three interprocess communication methods.
As a communication method, its performance is usually represented by its latency and throughput.
Latency is the time it takes from the point when a message is being passed to the operating system by the sender to send to the point when it is received by the receiver.
Throughput is the rate at which messages are being transmitted, i.e. how much bytes per second.
Usually when the size of the message is small, latency will dominate the total time spent on communication.
When the size of the message is large, size of the message divided by throughput will dominate the total time.

In our experiments, we will measure the time spent on the transmission of a message for each of the three fore-mentioned inter-process communication methods.
When size of the message is small, this measures latency.
When size of the message is large, this measures throughput.
For that purpose, we need to first confirm the accuracy of our clock by doing an experiment on the accuracy of available timing methods on Linux, which will be covered in Section~\ref{sec:clocks}.
As mentioned earlier, our observed latency and throughput will greatly depend on the size of the message.
Therefore we would identify the size of the message as a variable that will affect our measurements, and will measure our latencies and throughput on a set of different sizes of messages.

We shall next state our hypothesis about the results of our experiments.
We will discuss this from two perspectives: for a same communication method, how would the latency or throughput change with regard to the message size?
For a same message size, how would the latency or throughput change with regard to different communication methods?

\subsection{Expected Results with regard to Message Sizes}
When size of the message is small, the time spent is dominated by the system overhead of delivering a message, which is basically the latency.
Since system overhead will not change with the size of the message, we would estimate that our measured time spent will remain constant when message size is small.
When size of the message gets moderately bigger, system overhead no longer dominates, and the measured time spent should grow with the size of the message.
When the size of the message gets big enough, system overhead becomes negligible.
The major component of the time spent becomes the time needed to handle (i.e., copy the message from one place to another) the message, and it should be proportional to the length of the message.

\subsection{Expected Results with regard to Communication Methods}
In terms of latency, we would expect the latency of \texttt{mmap} to be the lowest.
The reason is that there is no system call involved when one is trying to write or read the shared memory, and the system overhead should be the smallest.
Inet socket should have the highest latency because every message transmitted via socket has to go though a complicated TCP protocol procedure.
In particular, according to the TCP protocol, every sent message must be replied by an ACK packet, which should significantly adds to its latency.

In terms of throughput, we would also expect \texttt{mmap} to be the best.
This is because in the shared memory model, the system does not have to copy the message.
However in \texttt{pipe} and socket, the message has to be copied from one buffer to another, and even possibly for several times (e.g., copy from send buffer to a system buffer, then from the system buffer to receive buffer).
In spite of this, \texttt{pipe} should still be better than socket, because of the added cost of the TCP protocol.

\section{Measuring the Clock Precision}
\label{sec:clocks}
Before measuring performance of communication methods, we will first perform experiments to select the best clock to use.
Since we will use the clock to measure fairly small time intervals, we want the clock to be as accurate as possible, so as to make sure our experiment results can be trusted.
However accuracy is not the only concern here.
Some accurate clock will not return a reliable timing for our use cases, as will be explained later.

On Linux, there are many ways to measure how much time spent during two operations.
In this section we will study the accuracy of three major ways of measuring time: \texttt{gettimeofday}, \texttt{clock\_gettime}, and x86 instruction \texttt{rdtsc}.
Both \texttt{gettimeofday} and \texttt{clock\_gettime} will return the current system time.
The difference is that \texttt{clock\_gettime} will provide better accuracy.
\texttt{rdtsc} is a x86 instruction that will return the value of a counter that how many cycles has passed since last reset.
To convert the cycle count to real time, one has to divide it by the frequency of the processor.
We would expect that \texttt{rdtsc} will return the most accurate timing.

\subsection{Methodology}
To test the accuracy of one timing method, we will measure the smallest number of operations that can be detected by the clock.
To be precise, we will try to insert some simple operations between two calls to our clock, and then increase the number of operations in between until the return values of the two calls are different.
If the return values are already different when there is no operations in between, then it means that the resolution of the clock is smaller than the overhead of making a single system call to it.

In our experiment, our test program tries to compute the series given by $a_n=a_{n-1}+1/a_{n-1}$.
We will first try two consecutive calls to our clock, and see if they would return the same time.
If so, we will then try to compute the first term of this series using the above recursion, and measure how much time is spent.
If our measured time is still $0$, we will then measure how much time is spent to compute the next $2$ terms.
We keep going until our measured time is nonzero when we try to compute $k$ terms, for some $k$.
We will use the value $k$ to indicate how accurate our clock is: the smaller $k$ is, the more accurate our clock is.

We coded our program in a way such that the computation will not use any loops, so as to minimize the number of instructions to be generated.
\footnote{This can be done neatly by using a C++ template trick.
Please refer to our code at \url{https://github.com/kirakira/cs736-mini-2/blob/master/clocks.cc}.}
In our program, one increment to $k$ corresponds to exactly $3$ added instructions: one division, one addition, and one move.
Though it is possible to make one computation even simpler, we must be very careful when doing it to avoid the computation becoming trivial, otherwise the optimizer will be clever enough to eliminate a lot of our code.
\footnote{One may argue here why not just turn off the optimizer.
In fact, the way we wrote our program took advantage of inline function optimizations provided by the optimizer.}

\subsection{Experiment Results}
Table~\ref{tab:1} summarizes our experiment results.
For each timer we conducted our experiment for several times, and computed the average of $k$ and its standard deviation.
The average stands for how accurate the timer is (smaller is more accurate), and the standard deviation stands for how reliable the timer is (smaller is more reliable).

We can see from the results that it takes on average $k=12$ for \texttt{gettimeofday} to detect the time difference.
For \texttt{clock\_gettime} and \texttt{rdtsc}, we always have $k=0$, which means both of them are more accurate than the limit of what we can measure in this experiment.

\subsection{Conclusion}
It is clear from the results that \texttt{clock\_gettime} and \texttt{rdtsc} are more accurate than \texttt{gettimeofday}.
Though in theory \texttt{rdtsc} should be more accurate than \texttt{clock\_gettime}, we are not able to verify this in our experiment.
However we can confirm that the resolution for \texttt{rdtsc} and \texttt{clock\_gettime} is smaller than the cost of making a function call, which is sufficient for our purposes.

Between the two choices, we eventually decided to use \texttt{clock\_gettime} as our clock.
The reason is that the value returned by \texttt{rdtsc} is processor-dependent: if our program was running on one CPU core and was switched to another core later, the \texttt{rdtsc} values returned by the $2$ cores are not comparable.
Unfortunately this will be the case when we perform our coming benchmark: when a program calls \texttt{read} it may block; when later it is ready to wake up it is subject to a context switch, and as a result the process may be put on a different core than the previous one by the operating system.
Therefore \texttt{rdtsc} does not work well with our benchmark.

\section{Methodology}
\label{sec:method}
In our experiments we want to measure the transmission time of messages of different sizes, i.e. from the point when ``send'' is called by the sender to the point when ``receive'' is returned to the receiver.
The set of message sizes we decide to use is $4$, $16$, $64$, $256$, $1$K, $4$K, $16$K, $64$K, $256$K, $512$K, $32$M, $64$M, $128$M, $256$M, $512$M, $1$G (unit in bytes).
Here up everything up to $512$K is supposed to measure the latencies, and starting from $32$M we want to use them to measure throughput.

Though the basic principles are the same no matter we want to measure latencies or throughput, our implementation techniques are slightly different.
\footnote{Our code is available at \url{https://github.com/kirakira/cs736-mini-2/blob/master/latency.cc}.}
In order to keep our clock readings as accurate as possible, we want to read our clock only on one side, i.e. only read in the parent process.
Hence after the child process has completed receiving the message, it must let its parent know by sending an acknowledgement so that the parent can stop its timer.
Time spent on the acknowledgement is non-negligible when the original transmission time is small, i.e. when we are measuring latencies.
Therefore when measuring latencies, instead of sending an acknowledgement, we will simply let the child process echo the message it has received so that the time measured by the parent will be the round trip time, and divide it by $2$ will give us an accurate latency.
When measuring throughput, we can just send a small acknowledgement message instead of echoing the entire message because the time spent on an acknowledgement is negligible compared to original transmission time.

The other thing we have noticed is the first communication always takes much longer than subsequent ones.
This is understandable because the operating system may decide to do extra initializations on the first communication.
Since the first communication involves a warm-up process that will not reflect its real performance, we decide not to include it towards our experiment results.

The workflows for measuring \texttt{pipe} and inet sockets are similar.
We first setup and initialize certain communication channels, either \texttt{pipe} or inet socket.
Then we call \texttt{fork} to spawn the child process, and after that we are ready to communicate.
For each message, the parent will first start the timer, send the message through calling \texttt{write}, and wait for the response from child by calling \texttt{read}, and stop the timer; in the meanwhile, the child will first call \texttt{read} to wait for parent's message to be delivered, then depending on whether it is a latency test or a throughput test, it will then call \texttt{write} to either echo the entire message or simply write an acknowledgement.

Measuring \texttt{mmap} is a bit different.
To send a message we will simply call \texttt{memcpy} to copy our message to the shared memory.
However we must make use of additional synchronization mechanisms to notify the completion of \texttt{memcpy}.
In our implementation we decide to use a shared condition variable coupled with a shared state variable.
There are $3$ possible states during each measurement: $(0)$ parent sending message to child, $(1)$ child sending reply to parent, and $(2)$ parent finalizing measurement (i.e. to stop the timer).
Initially we are in state $0$ and the parent is awake and the child should sleep on the condition variable.
At this moment parent shall start the timer and copy the message to shared memory.
After copying has finished, parent change state variable to $1$, and signal the condition variable, and finally put itself into sleep on the condition variable.
The child will then wake up, and make a reply to the message, and change state to $2$, and signal the condition variable, and put itself to sleep on the condition variable.
Then parent would wake up again, stop the timer, and reset the state to $0$ for the next round of message transmission.

An alternative way to using a condition variable is to just use a mutex, and do polling on the state variable instead of sleeping and waiting for being waken.
We think this is worse than our conditional variable approach because having parent and child competing for a mutex to do polling could possibly waste them a lot of CPU cycles, and thus making our measurements inaccurate.

\section{Results and Conclusions}
\label{sec:results}
Figure~\ref{fig:1} shows our test results for latencies, and Figure~\ref{fig:2} shows the results for throughput.


% references section

% can use a bibliography generated by BibTeX as a .bbl file
% BibTeX documentation can be easily obtained at:
% http://www.ctan.org/tex-archive/biblio/bibtex/contrib/doc/
% The IEEEtran BibTeX style support page is at:
% http://www.michaelshell.org/tex/ieeetran/bibtex/
%\bibliographystyle{IEEEtran}
% argument is your BibTeX string definitions and bibliography database(s)
%\bibliography{IEEEabrv,../bib/paper}
%
% <OR> manually copy in the resultant .bbl file
% set second argument of \begin to the number of references
% (used to reserve space for the reference number labels box)
\begin{thebibliography}{1}

\bibitem{IEEEhowto:kopka}
H.~Kopka and P.~W. Daly, \emph{A Guide to \LaTeX}, 3rd~ed.\hskip 1em plus
  0.5em minus 0.4em\relax Harlow, England: Addison-Wesley, 1999.

\end{thebibliography}




% that's all folks
\end{document}


